\section{相关工作}

%wsq 参考学术paper的相关工作格式,将相关工作分为几类介绍。
传统LLM推理框架采取了很多显存优化技术。

\subsubsection{早期工作}

Hugging Face Accelerate~\cite{Huggingface-Accelerate}实现了张量交换技术,但换出与换入的张量仅限于LLM的参数张量。DeepSpeed ZeRO-Inference~\cite{GPT-175B资源消耗}实现了LLM的分布式推理,能够利用数据并行性与张量并行性来实现LLM推理加速。但以上两个框架均无法针对KV Cache实现张量交换技术,也没有实现张量重算。
  
\subsubsection{张量交换的提出}

FlexGen~\cite{Swapping}首次提出了“自适应内存优化”的概念,并将张量交换的范围由参数张量扩展至所有张量。通过线性规划建模在交换方案的可行域内进行搜索,在给定的时间内找到较优解。同时,FlexGen还实现了张量压缩技术,相比于Hugging Face Accelerate和DeepSpeed ZeRO-Inference,实现了较大的吞吐率提升。然而,FlexGen假设运行队列中的所有用户请求拥有相同的输出长度。在实际情况下,输出长度具有很大的差异性,使得相关理论无法推广。
  
\subsubsection{调度粒度的转变}

ORCA~\cite{ORCA}将批处理调度的粒度从单个用户请求转化为单次推理迭代,化解了用户请求相互等待的性能瓶颈。vLLM~\cite{vLLM}在ORCA的基础上实现了张量重算技术。基于OS页式内存管理思想,引入Paged Attention机制来实现。vLLM相比于OCRA,大幅度提升显存利用率,并增加批处理大小上限,进而提升推理任务的整体吞吐率。同时,vLLM设计了规范且友好的用户接口,开发者能够根据任务需求来定义各种参数,极大地方便了有关推理优化的深入研究。
  
\subsubsection{投机推理技术}

SpecInfer~\cite{SpecInfer}引入了投机推理技术(Speculative Sampling),根据小型LLM的输出来预测大型LLM的输出,在大幅度提升推理吞吐率的同时保障了输出质量。DistillSpec~\cite{DistillSpec}在SpecInfer的基础上实现了知识蒸馏技术(Knowledge Distillation,KD),使得输出预测准确率显著提升。
