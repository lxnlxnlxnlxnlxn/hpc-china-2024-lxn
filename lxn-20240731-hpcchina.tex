%! TeX program=latexmk
%! TeX options=-xelatex -synctex=1 -interaction=nonstopmode -file-line-error "%DOC%"
\documentclass[nosysfonts, a4paper]{hpcchina}

\usepackage{graphicx}
\usepackage{amsmath,amsthm}
\usepackage{amssymb,amsfonts}
%以下宏包为测试用途
\usepackage{blindtext}
\usepackage{zhlipsum}
\usepackage{tikz}
\usepackage{metalogo}

%LXN
\graphicspath{ {pictures/} }  %使用图片
\usepackage{algorithm}  % 伪代码编辑
\usepackage{algorithmic}  % 伪代码编辑
\renewcommand{\algorithmicrequire}{\textbf{Input:}}  % 伪代码输入
\renewcommand{\algorithmicensure}{\textbf{Output:}}  % 伪代码输出
\usepackage{setspace}  % 伪代码行距
\usepackage{booktabs}
\usepackage[UTF8]{ctex}

\tracinglostchars=2

%标题
\title{AdaptiveLLM:基于自适应张量交换和张量重算的大语言模型推理优化}

%作者
\author{
  梁绪宁\textsuperscript{1,*},
  王思琪\textsuperscript{1,*},
  杨海龙\textsuperscript{1},
  栾钟治\textsuperscript{1},
  刘轶\textsuperscript{1},
  钱德沛\textsuperscript{1}
}
%单位
\affiliation{
  \textsuperscript{1} 北京航空航天大学,北京 100191
}
%邮箱,使用\mailurl{地址}可以生成可点击的链接
\email{
  \mailurl{
    liangxuning@126.com,
    lethean1@buaa.edu.cn,
    hailong.yang@buaa.edu.cn,
    07680@buaa.edu.cn,
    yi.liu@buaa.edu.cn,
    depeiq@buaa.edu.cn
  }
}

%中文摘要
\cabstract{
  大语言模型(LLMs)拥有极高的参数量,为推理任务带来GPU内存瓶颈。已有LLM推理框架引入张量交换和张量重算等内存优化技术,在有限的GPU内存上通过牺牲性能完成推理。然而,已有工作无法根据推理任务运行时信息自适应地选择内存优化技术,导致推理任务的性能无法进一步提升。同时,这些工作大都以推理任务整体吞吐率为单一优化目标,缺乏对用户请求调度的公平性考虑。针对以上问题,本文面向LLM推理服务场景,提出了AdaptiveLLM,一款基于自适应张量交换和张量重算的LLM推理服务框架。AdaptiveLLM实现了张量重算和张量交换开销精准预测,其预测误差分别控制在2\%和4\%以下。在此基础上,引入基于开销感知的内存优化策略,可以自适应地选择开销较低的内存优化技术,提高任务整体吞吐率。同时,引入基于公平性的用户请求调度策略,降低单请求延时。本文在主流LLM模型和数据集上开展实验验证,以vLLM和DeepSpeed作为基准程序进行对比评估。结果表明,AdaptiveLLM分别实现了$1.3\times\sim2.1\times$和$1.4\times\sim2.3\times$的整体吞吐率提升,同时使平均带权周转时间降低,压缩比例分别为$20\%\sim50\%$和$25\%\sim65\%$。由此证明AdaptiveLLM在推理优化过程中可以更好地权衡整体吞吐率与用户请求调度公平性,实现LLM高效推理。
}
%中文关键词
\keyword{
  大语言模型;推理;张量交换;张量重算
}

%英文标题
\etitle{AdaptiveLLM: Efficient LLM inference using adaptive tensor swapping and re-computation techniques}
%英文作者
\eauthor{
  Liang Xuning\textsuperscript{1},
  Wang Siqi\textsuperscript{1},
  Yang Hailong\textsuperscript{1},
  Luan Zhongzhi\textsuperscript{1},
  Liu Yi\textsuperscript{1},
  Qian Depei\textsuperscript{1}
}
%英文单位
\eaffiliation{
  \textsuperscript{1} (Beihang University, Beijing 100191)
}
%英文摘要
\eabstract{
  Large Language Models(LLMs) come with an extremely high amount of parameters, posing significant challenges on inference tasks. Traditional LLM inference services employ tensor swapping and tensor re-computation techniques, guaranteeing the success of generation at the cost of performance on limited GPU memory. However, existing LLM serving systems fail to search memory management schemes adaptively based on runtime information, leading to a sub-optimal performance. Furthermore, these works focus exclusively on the throughput of the inference tasks, lacking consideration for scheduling fairness. To address the above issues, we propose \textit{AdaptiveLLM}, an efficient LLM serving framework for inference tasks based on swapping and re-computation. Specifically, \textit{AdaptiveLLM} implements an overhead predictor for swapping and re-computation, with an error rate lower than 2\% and 4\% respectively. \textit{AdaptiveLLM} also adopts a cost-aware memory optimization algorithm which improves throughput on the server, and a fairness-based request scheduling algorithm which reduces the proportion of waiting time for the client. On typical LLMs and datasets, \textit{AdaptiveLLM} achieves a speedup by $1.3\times\sim2.1\times$ and $1.4\times\sim2.3\times$, while reduces the average weighted around time by $20\%\sim50\%$ and $25\%\sim65\%$, compared with the \textit{vLLM} and \textit{DeepSpeed} baseline. In conclusion, \textit{AdaptiveLLM} achieves efficient LLM inference by making a better trade off between throughput and fairness.
}
%英文关键词
\ekeyword{
  LLM; Inference; Tensor Swapping; Tensor Re-computation
}

%基金号
\grants{国家重点研发项目(2023YFB3001801), 自然科学基金项目(62322201, 62072018, U23B2020, U22A2028), 中央高校基本科研业务费专项资金资助(YWF-23-L-1121, JKF-20240198), 复杂软件全国重点实验室(SKLSDE-2023ZX-05) \\ 通信作者:杨海龙}
%DOI号
\doi{missing DOI}
%分类号
\clcls{TP391}
%卷(期):起止页,年
\issue{卷(期):起止页,年}
%收稿日期
\dateaccept{2024-07-31}
%修回日期
\daterevise{2024-08-31}

\begin{document}
  \maketitle
  \section{引言}

从人脸识别~\cite{Face-Recognition}、个性化推荐~\cite{Personal-Recommendation}、到智能家居~\cite{Smart-Home}、无人驾驶~\cite{Self-Driving}等应用领域,深度学习(Deep Learning, DL)~\cite{Deep-Learning}相关技术已经融入到社会的方方面面,为人类的生产生活带来了极大的便利。自然语言处理(Natural Language Processing, NLP)~\cite{NLP}作为深度学习领域的重要研究方向,长期以来备受研究人员关注。近年来,随着GPU算力的不断提升,各种语言模型也朝着更大参数量、更高准确度的方向迅猛发展。

大语言模型~\cite{LLM}(Large Language Models,LLM)是自然语言处理领域的一个分支。LLM通常拥有十亿级别,甚至万亿级别的参数量,因此需要海量文本数据进行训练。同时,LLM在多种类型的任务中展现出卓越性能,如文本摘要~\cite{Text-Summarization}、机器翻译~\cite{Machine-Translation}、代码生成~\cite{Code-Generation}、以及对话问答~\cite{Question-Answer}等,拥有巨大的科研潜力与商业价值。2021年GPT-3模型~\cite{Text-Summarization, GPT3}的问世标志着LLM研究领域的一个里程碑,自此,各大科研机构纷纷投入到相关研究中,各种LLM层出不穷,使得该领域的研究和应用热度空前高涨。

复杂的模型结构和庞大的参数量为LLM带来了卓越的应用效果,但却为LLM部署后的推理性能优化带来了极大挑战。特别地,LLM庞大的参数量导致推理过程中产生极高的内存占用。例如,GPT-175B模型仅在权重加载环节就需要消耗325GB的GPU内存空间~\cite{GPT-175B资源消耗},需要使用至少5个NVIDIA A100 GPU(80GB),并引入复杂的推理并行化策略才能够完成模型推理。因此,如何降低LLM推理任务的内存资源占用对于LLM成功部署和推广至关重要。

传统的深度学习模型研究中提出了张量交换~\cite{Swapping}(Swapping)、张量重算~\cite{Recomputation}(Recomputation)等技术来降低推理过程中的显存占用。具体而言,张量交换技术通过张量生命周期分析、异步传输、动态调度等机制,将推理过程中不需要立即使用的张量从GPU内存交换到CPU内存;而张量重算则是将推理过程中的部分中间张量在不需要时释放,将计算图的关键节点保存下来,并在需要时重新计算这些中间结果。然而,简单地将张量交换和张量重算技术应用于LLM推理框架会导致以下两点不足:

首先,张量交换和张量重算技术虽然可以降低推理过程的GPU显存资源占用,但其对LLM推理性能的影响十分复杂,取决于服务器硬件配置(如GPU的计算能力,CPU-GPU双向传输带宽)、用户设置(如生成新token时的采样方式)、LLM任务类型与数据集选取、以及推理任务的运行时信息等。已有的LLM推理框架~\cite{Swapping, vLLM, ORCA}虽然已经集成了张量交换或张量重算技术,但其在GPU显存不足时只能固定选择上述技术中的一种,而无法根据上述信息选择更优者。\textcolor{red}{因此,寻找一种面对GPU内存不足时,能够在张量交换和张量重算间进行自适应选择,并进行合理调度的LLM推理服务框架,就显得尤为重要。}

% 另外,单请求延时和整体吞吐率在优化过程中存在矛盾。一部分传统工作,如FasterTransformer~\cite{FasterTransformer}等,以单请求延时为单一优化目标,将大部分集群资源分配给当前运行的少数请求,使得集群资源利用率降低,限制整体吞吐率;另一部分工作,如ORCA~\cite{vLLM, ORCA}等,以整体吞吐率为单一优化目标,通过提升批处理大小来增加资源利用率,但会导致单请求时延增加。整体吞吐率是面向服务器端的性能优化指标,体现了服务器端的处理效率。单请求平均延时是面向客户端的性能优化指标,体现了用户请求处理的实时性。在使用张量交换或张量重算技术优化LLM推理任务显存占用时,需要考虑其对推理任务整体吞吐率和单请求延时的影响,并在二者间进行权衡。

\textcolor{red}{然而,大部分传统工作在调度时忽略了公平性因素。例如在vLLM中,张量交换与张量重算拥有不同的优先级设置方案,导致不同用户请求的等待时间存在显著差异。这些工作大都以整体吞吐率作为性能优化指标,体现服务器端的处理效率。为了面向客户端体现请求间的调度公平性,本文引入平均带权周转时间作为度量指标,来反映用户请求的等待时间平均占比。综上所述,在使用张量交换或张量重算技术优化LLM推理任务显存占用时,需要考虑其对推理任务整体吞吐率和用户请求间公平性的影响,并在二者间进行分析权衡。}

为了解决上述缺陷,本文提出了AdaptiveLLM,一个基于张量交换和张量重算的自适应LLM推理服务框架。该框架实现了针对张量交换和张量重算的精准开销分析,并调用基于开销感知的内存优化策略和基于公平性的用户请求调度策略,在张量交换和张量重算技术间进行动态选择,在降低LLM推理显存占用的同时,降低其对LLM推理吞吐量和用户请求公平性的影响,进而实现LLM任务的高效推理。

具体而言,本文开展了以下工作:

\begin{itemize}

    \item \textbf{本文设计了一款张量重算开销分析器,实现了张量重算开销的精准预测。}通过算子粒度计算复杂度分析来识别张量重算开销的影响因素,而后建立回归预测模型预测单步推理执行时间。实验表明,张量重算开销分析器的预测误差在2\%以内。
    
    \item \textbf{本文设计了一款张量交换开销分析器,实现了张量交换开销的精准预测。}本文获取用户请求KV Cache的内存占用和GPU-CPU间通信效率信息,来对张量交换的数据传输开销进行预测。实验表明,张量交换开销分析器的预测误差在4\%以内。

    \item \textbf{本文设计和实现了一个基于张量交换和张量重算的自适应LLM推理服务框架AdaptiveLLM。}本文引入基于开销感知的内存优化策略,动态选择相应的内存优化技术,通过降低显存占用,来提升推理任务的整体吞吐率。\textcolor{red}{同时引入基于公平性的用户请求调度策略,降低带权周转时间。}
    
    \item \textbf{本文选择典型LLM模型和数据集对AdaptiveLLM进行全面实验评估。} 在典型LLM模型(OPT~\cite{OPT}、Llama~\cite{Llama})和数据集(Chatbot~\cite{Chatbot}、Alpaca~\cite{Alpaca}、Summary~\cite{Summary})上对AdaptiveLLM的张量重算开销分析器、张量交换开销分析器、内存优化决策器、和用户请求调度器模块的有效性进行了实验验证。结果表明,以vLLM和DeepSpeed作为基准框架时,AdaptiveLLM能够分别实现$1.3\times\sim2.1\times$和1.4$\times\sim2.3\times$的整体吞吐加速比,同时降低用户请求平均带权周转时间,缩减比例分别为$20\%\sim50\%$和$25\%\sim65\%$。

\end{itemize}
  \section{背景知识}

本章介绍有关AdaptiveLLM的背景知识。由于AdaptiveLLM主要面向LLM推理过程中产生的KV Cache内存占用进行优化,因此本章将在第一节阐明KV Cache在LLM推理任务中的功能,在第二节论述传统工作中面向KV Cache的内存优化技术。

\subsection{KV Cache的提出}

LLM推理任务以token作为输入与输出的基本单位。对于生成式推理任务,每次前向传播计算仅生成一个新token。一般来说,其包含两个阶段:prefill阶段读取用户输入的token序列,生成第一个token;decode阶段分为多步进行,依次生成后续token,直至得到终止token。在推理过程中,每个token拥有一个key-value张量对,为自注意力机制下的编码结果。

在decode阶段中,每个token的计算均依赖于前序token的key值和value值。如果每次计算前都重新调用自注意力机制来获取前序token的key-value张量,则会产生大量不必要的计算开销。主流LLM推理服务~\cite{Swapping, vLLM, ORCA, SpecInfer}框架普遍采用KV Cache数据结构来保存这些token的key-value张量,方便后续token的生成,避免重复计算,其工作原理如图\ref{Fig:KV Cache的功能示意图}所示。

\begin{figure}[!htbp]
  \centering
  \includegraphics[width=0.9\linewidth]{KV Cache的功能示意图.png}
  \caption{KV Cache的功能示意图}
  \label{Fig:KV Cache的功能示意图}
\end{figure}

然而,随着后续token的不断生成,KV Cache迅速扩展,产生推理内存瓶颈。例如,在OPT-13B模型中,对于一个长度为100的用户请求,其KV Cache能够占用39.1MB的内存空间。有限的GPU内存将批处理大小限制在较低水平,阻碍推理并发度的进一步提升,进而限制吞吐率。

不同于LLM参数张量,KV Cache占用的内存空间在对应用户请求推理完毕后被释放。 其内存占用量大、动态性高,拥有较大的优化空间,因此AdaptiveLLM的内存优化策略将针对KV Cache实现。

\subsection{KV Cache的内存优化}

KV Cache的引入方便了计算过程,却带来内存瓶颈,使得LLM推理性能的提升无法达到预期水平。下面介绍针对KV Cache内存占用的一些优化工作。

\subsubsection{内存碎片优化}

在传统LLM推理服务框架~\cite{Swapping}中,内存管理器按照用户定义的序列长度上限,为每个请求设置一块固定大小的GPU内存来存储KV Cache。但用户请求长度的差异性导致内碎片的大量产生。为了解决该问题,部分LLM推理框架~\cite{Output-Length-Prediction}能够基于历史信息来预测输出长度,并按照预测值分配内存。然而,预测误差会导致输出截断,且旧请求的完成与新请求的加入使得内存中产生很多外碎片。随着新请求的不断到来,内碎片与外碎片在内存中积累,严重影响了内存空间的高效使用。基于这些问题,vLLM框架~\cite{vLLM}引入了Paged Attention机制,基于OS页式内存管理思想,将GPU内存划分成块,并通过维护块表来支持KV Cache在内存空间中的不连续存储。该机制基本消除了内碎片和外碎片现象,大大提升内存利用率。

\subsubsection{张量交换与张量重算}

为了攻克推理内存瓶颈,传统框架引入了张量交换技术~\cite{Swapping, vLLM, LightLLM},将暂时不会使用的KV Cache传输至CPU中,在计算需要时重新传输至GPU中。然而,CPU-GPU间有限的PCIe带宽使得换出和换入过程产生不可忽略的通信开销,限制吞吐率,降低推理性能。部分研究提出~\cite{Recomputation},当张量交换带来的开销超过重新调用自注意力机制的开销时,应选择后者来获取所有前序token的key-value张量,也称张量重算。具体来说,内存管理器直接删除重算请求对应的KV Cache,在其被调度时执行一次prefill阶段来代替原本应该执行的decode阶段。重算与交换的联合使用缓解了通信开销问题,然而,当GPU内存不足时,如何在二者中进行选择成为了新的困境。AdaptiveLLM针对此问题设计了基于开销感知的内存优化策略,能够预测二者的开销,并选择开销小的过程执行。

% wsq 把这一章融合进上一节中,不用提张量压缩
% \subsection{针对KV Cache的张量优化技术}

% 在LLM推理服务过程中,传统的张量优化(也称抢占)技术有三种:张量交换、张量重算和张量压缩~\cite{Swapping}。AdaptiveLLM实现了张量交换与张量重算。而张量压缩目前还未能实现,将在本文第五章介绍。

% \subsubsection{张量交换}

% 服务器拥有GPU-CPU-磁盘三级存储结构。GPU位于三级存储中的最上层,其计算速率快,并行度高,但存储空间有限,而CPU和磁盘的存储空间相对较大。为了提升服务器的实时吞吐率,LLM推理服务框架一般采用批处理的方式执行用户请求。随着批处理大小的增加或模型参数量的扩展,运行时需要保存的张量会超出GPU的内存限制。当检测到GPU内存占用峰值达到较高水平时,需要开启张量交换功能,将一部分需要保存,而暂时用不到的张量换出到CPU甚至磁盘中,在计算需要时重新换入GPU中。综上所述,张量交换包括换出与换入两个阶段,有两次数据传输过程。

% \subsubsection{张量重算}

% 在抢占式用户请求调度系统中,当某个请求获得执行权时,会检查之前的计算结果是否保存在GPU中,如果不在,则需要重新获取这部分计算结果。此时可以无需将之前存储的计算结果(如果有)从CPU或磁盘中换入到GPU中,而仅仅对它们进行重新计算。对于执行LLM推理任务的用户请求而言,这些key-value张量在初次生成时经历了多次前向传播,而在重算过程中仅需调用自注意力机制即可得到,因此张量重算的开销远远小于token序列初次生成时的开销,不会导致计算量的爆炸式增长。
  % % wsq 这应该算是motivation实验,基于这个观察你才有空间去选择策略。不应该出现在大实验中。
\section{研究意义}

本文选用OPT-13B、OPT-30B、Llama-13B和Llama-32.5B进行实验,针对单个用户请求,测试张量交换与张量重算开销随序列长度的变化关系。其结果如图\ref{Fig:交换与重算开销对比}所示。当序列长度较小时,交换开销小于重算开销。随着序列长度的增加,二者大小关系反转。

\begin{figure}[!htbp]
    \centering
    \includegraphics[width=0.85\linewidth]{交换与重算开销对比.png}
    \caption{交换与重算开销对比}
    \label{Fig:交换与重算开销对比}
  \end{figure}

\begin{itemize}    
    \item \textbf{张量重算开销}:自注意力机制内核采用并行计算策略,每个线程只计算一个token的qkv张量及注意力值。随着序列长度的增长,线程数量增加,同步开销随之上升,而单线程计算量不变,导致张量重算开销随序列长度增加呈亚线性增长。
    \item \textbf{张量交换开销}:KV Cache保存每个token的key-value张量,其内存占用与序列长度呈正比关系,而PCIe传输带宽在推理过程中基本保持稳定。因此张量交换开销与序列长度也呈正比关系。
\end{itemize}

因此,张量重算开销随序列长度的增长速度小于张量交换开销。在贪心采样策略下,对于长序列而言,无论是vLLM还是AdaptiveLLM,都使用张量重算,两种策略带来内存优化行为没有差异。对于短序列而言,vLLM使用张量重算,而AdaptiveLLM使用开销较小的张量交换,此时能够带来整体吞吐率提升。且OPT-13B和Llama-13B相比于OPT-30B和Llama-32.5B,在序列长度较短时,张量交换相比于张量重算,开销优势更加明显。

在LLM实际应用场景中,大多数序列的长度较短,使得张量交换在提升性能上拥有明显优势。而当长序列较多,或者CPU内存空间不足时,张量重算技术能够发挥优势。AdaptiveLLM能够基于服务器软硬件环境、LLM与数据集选取、用户参数设置与推理任务运行时信息来精准预测张量交换与张量重算开销,并选择开销较小的内存优化技术执行。 

  \begin{figure*}[!htbp]
    \centering
    \includegraphics[width=1\linewidth]{张量交换示意图.png}
    \caption{张量交换示意图} 
    \label{Fig:张量交换示意图}
  \end{figure*}

\section{相关工作}

\subsubsection{张量交换技术}

随着批处理大小的增加或模型参数量的扩展,运行时需要保存的张量会超出GPU内存限制。张量交换技术在GPU空间不足时开启,将一部分需要保存,而暂时用不到的张量换出至CPU中,在计算需要时重新换入GPU中。图\ref{Fig:张量交换示意图}展示了张量交换的基本流程。

HuggingFace Accelerate~\cite{Huggingface-Accelerate}实现了张量交换技术,但换出与换入的张量仅限于LLM的参数张量。LightLLM~\cite{LightLLM}能够针对KV Cache进行张量交换,但换出的比例设计为定值,无法根据运行时信息调整。

FlexGen~\cite{Swapping}首次提出了“自适应内存优化”的概念,通过线性规划建模在交换方案的可行域内进行搜索,在给定的时间限制内找到一个较优解。然而,FlexGen假设运行队列中的所有用户请求拥有相同的输出长度。在实际情况下,输出长度具有很大的差异性,使得相关理论无法推广。

本文针对KV Cache实现张量交换技术,并进行细粒度内存占用建模分析。根据GPU内存使用水平,进行实时换出换入调整。

\subsubsection{张量重算技术}

Mimose~\cite{Recomputation, Recomp_2, Recomp_3}等工作提出了张量重算技术。在抢占式用户请求调度系统中,当某个请求获得执行权时,会检查之前的计算结果是否保存在GPU中,如果不在,则需要重新获取这部分计算结果。图\ref{Fig:张量重算示意图}展示了张量重算的基本流程。

\begin{figure}[!htbp]
  \centering
  \includegraphics[width=1\linewidth]{张量重算示意图.png}
  \caption{张量重算示意图} 
  \label{Fig:张量重算示意图}
\end{figure}

张量重算开销的计算相比于张量交换略微复杂。Capuchin~\cite{Capuchin}将张量重算开销计算过程分解到算子粒度。对于每个算子,通过记录其输入张量与输出张量的生成时间,来获取该算子的重算开销。AdaPipe~\cite{AdaPipe}将连续出现的多个算子组合成计算单元,通过模拟运行来记录各个计算单元的重算开销。

本文针对KV Cache实现张量重算技术。这些key-value张量在初次生成时经历了多次前向传播,而在重算过程中仅需调用自注意力机制即可得到,因此张量重算的开销远远小于token序列初次生成时的开销,不会导致计算量的爆炸式增长。

\subsubsection{LLM推理优化技术}

除了张量交换和张量重算等针对张量层面的优化策略以外,传统LLM推理服务框架还采用了很多其它的推理优化技术。

ORCA~\cite{ORCA}将批处理调度的粒度从单个用户请求转化为单次推理迭代,化解了用户请求相互等待的性能瓶颈。vLLM~\cite{vLLM}基于OS页式内存管理思想,在ORCA的基础上引入Paged Attention机制。vLLM相比于OCRA,大幅度提升显存利用率,增加批处理大小上限,进而提升推理任务的整体吞吐率。

SpecInfer~\cite{SpecInfer}引入了投机推理技术(Speculative Sampling),根据小型LLM模型的输出来预测大型LLM模型的输出,在大幅度提升推理吞吐率的同时保障了输出质量。DistillSpec~\cite{DistillSpec}在SpecInfer的基础上实现了知识蒸馏技术(Knowledge Distillation,KD),使得输出预测的准确率显著提升。

本文提出的LLM推理优化策略能够与调度粒度转化、投机推理等研究工作相兼容。本文设计了规范且友好的用户接口,开发者能够根据任务需求来定义各种参数,极大地方便了有关推理优化的深入研究。

  \input{4.design}
  \section{实验验证}

本章介绍实验部分。第一节为实验平台软硬件配置。第二节介绍LLM模型与数据集选取,以及实验参数设置。第三节针对基于开销感知的内存优化策略,进行吞吐率测试。第四节针对基于公平性的用户请求调度策略,进行公平性测试。第五节分析张量交换与张量重算预测误差。第六节进行开销测试。

\subsection{实验环境}

本文开展实验使用的服务器软硬件配置如表\ref{Table:实验平台的软硬件配置}所示。使用Intel(R) Xeon(R) CPU和NVIDIA A800 80GB GPU作为硬件环境,使用CUDA-11.8、PyTorch-2.0.l、Ray2.7.1以及vLLM-0.2.5作为底层框架进行开发。服务器使用PCIe连接实现GPU-CPU通信。

\begin{table}[H]
  \centering
  \caption{实验平台的软硬件配置}
  \label{Table:实验平台的软硬件配置}
  \renewcommand{\arraystretch}{1.25}
  \small
  \begin{tabular}{c c}
    \toprule
    \textbf{软件/硬件} & \textbf{型号/版本} \\ 
    \midrule
    CPU & Intel(R) Xeon(R) CPU @ 2.60GHz  \\ 
    GPU & NVIDIA A800 PCIE 80GB \\ 
    OS & CentOS Linux 7 (Core) \\ 
    CUDA & 11.8 \\ 
    PyTorch & 2.0.1 \\ 
    Ray & 2.7.1 \\
    vLLM & 0.2.5 \\ 
    \bottomrule
  \end{tabular}
\end{table}

\subsection{模型与数据集}

本文选用OPT~\cite{OPT}(OPT-13B、OPT-30B)和Llama~\cite{Llama}(Llama-13B、Llama-32.5B)作为实验模型,在三个常见数据集(Chatbot~\cite{Chatbot}、Alpaca~\cite{Alpaca}、Summary~\cite{Summary})上进行测试。数据集信息如表\ref{Table:实验数据集选取}所示。

\begin{table}[H]
  \centering
  \caption{实验数据集选取}
  \label{Table:实验数据集选取}
  \renewcommand{\arraystretch}{1.25}
  \small
  \begin{tabular}{c c c c}
    \toprule
    \textbf{数据集} & \textbf{样本总数} & \textbf{平均输入长度} & \textbf{任务类型} \\
    \midrule
    Chatbot & 258064 & 17.02 & 对话类 \\
    Alpaca & 68912 & 19.66 & 指令类 \\
    Summary & 1799 & 340.48 & 摘要类 \\
    \bottomrule
  \end{tabular}
\end{table}

\begin{figure}[!htbp]
  \centering
  \includegraphics[width=0.9\linewidth]{序列长度分布曲线.png}
  \caption{序列长度分布曲线}
  \label{Fig:序列长度分布曲线}
\end{figure}

三个数据集的样本序列长度分布曲线如图\ref{Fig:序列长度分布曲线}所示。Chatbot和Alpaca中大多数序列长度较短,而Summary中序列长度展现出很大差异性,且包含长序列。它们涵盖了LLM应用程序面临的大部分场景。

实验过程中,将GPU内存与CPU内存使用率限制在较低水平,触发用户请求抢占现象,以模拟LLM应用程序在多任务并发场景下的运行状态。针对12个实验组,5.3章节使用整个数据集进行吞吐率测试,5.4章节在相应数据集中使用简单随机抽样法选取1000个样本进行公平性测试。

\begin{figure*}[!htbp]
  \centering
  \includegraphics[width=0.9\linewidth]{整体吞吐率.png}
  \caption{推理任务吞吐率}
  \label{Fig:推理任务吞吐率}
\end{figure*}

\subsection{吞吐率测试}

本文以vLLM和DeepSpeed-MII~\cite{DeepSpeed-MII}作为基准框架,针对AdaptiveLLM进行吞吐率测试。

\begin{itemize}

  \item \textbf{vLLM:}4.4章节已经讲述,vLLM在GPU内存不足时使用单一的张量优化策略,由于本文在推理过程中使用贪心采样方式来生成新token,因此vLLM内存管理器固定调用张量重算技术。
  
  \item \textbf{vLLM$\_$s:}对vLLM框架稍加修改形成vLLM$\_$s,使其在GPU内存不足时固定调用张量交换技术。
  
  \item \textbf{DeepSpeed-MII:}DeepSpeed~\cite{DeepSpeed}是Microsoft推出的一系列LLM高效训练或推理服务框架。DeepSpeed-MII集成了分块KV Cache缓存~\cite{vLLM}、连续批处理~\cite{Continuous-Batching}、动态分割融合~\cite{DeepSpeed-FastGen}等多种LLM推理优化技术,显著降低LLM推理成本。但DeepSpeed-MII缺少对张量交换技术的支持,且在用户请求调度过程中并未考虑公平性因素。
  
\end{itemize}

\begin{table}[H]
  \centering
  \caption{AdaptiveLLM相对于基准框架的加速比}
  \label{Table:AdaptiveLLM相对于基准框架的加速比}
  \renewcommand{\arraystretch}{1.25}
  \small
  \begin{tabular}{c c c}
    \toprule
    \textbf{LLM} & \textbf{dataset} & \textbf{vLLM/vLLM\_s/DeepSpeed-MII} \\
    \midrule
    OPT-13B	& alpaca & 1.72$\times$/2.17$\times$/1.87$\times$ \\
    OPT-13B	& chatbot & 1.56$\times$/2.15$\times$/2.24$\times$ \\
    OPT-13B	& summary & 1.31$\times$/1.72$\times$/1.97$\times$ \\
    OPT-30B	& alpaca & 1.63$\times$/2.10$\times$/2.11$\times$ \\
    OPT-30B	& chatbot & 1.94$\times$/2.06$\times$/1.88$\times$ \\
    OPT-30B	& summary & 1.39$\times$/1.71$\times$/2.08$\times$ \\
    Llama-13B & alpaca & 1.61$\times$/2.00$\times$/1.83$\times$ \\
    Llama-13B & chatbot & 1.73$\times$/2.13$\times$/1.96$\times$ \\
    Llama-13B & summary & 1.42$\times$/2.08$\times$/1.70$\times$ \\
    Llama-32.5B & alpaca & 1.80$\times$/2.04$\times$/1.89$\times$ \\
    Llama-32.5B & chatbot & 2.06$\times$/2.60$\times$/1.47$\times$ \\
    Llama-32.5B & summary & 1.36$\times$/1.66$\times$/1.66$\times$ \\
    \bottomrule
  \end{tabular}
\end{table}

图\ref{Fig:推理任务吞吐率}展示了12个实验组在推理任务中的整体吞吐率测试结果,其横坐标为单序列最大输出长度。表\ref{Table:AdaptiveLLM相对于基准框架的加速比}给出了单序列最大输出长度为64时,AdaptiveLLM相对于三个基准框架的加速比。结果表明,相比于vLLM、vLLM$\_$s和DeepSpeed-MII基准框架,AdaptiveLLM实现了$1.3\times\sim2.1\times$,$1.6\times\sim2.6\times$,以及$1.4\times\sim2.3\times$的整体吞吐加速比。究其原因,在GPU内存不足时,vLLM和DeepSpeed固定调用张量重算技术,vLLM$\_$s固定调用张量交换技术,三者均无法通过预测张量交换和张量重算开销来选择更优者。若当前可使用的GPU内存较少,其性能会显著低于拥有自适应张量优化策略的AdaptiveLLM。另外,若换出的请求较多,需要考虑CPU内存不足的情况,vLLM$\_$s的批处理大小也因此被限制在较低水平。

\begin{table}[H]
  \centering
  \caption{推理任务抢占行为记录}
  \label{Table:推理任务抢占行为记录}
  \renewcommand{\arraystretch}{1.25}
  \small
  \begin{tabular}{c c c c}
    \toprule
    \textbf{LLM} & \textbf{数据集} & \textbf{重算次数} & \textbf{交换次数} \\
    \midrule
    OPT-13B & alpaca & 1464 & 91 \\
    OPT-13B & chatbot & 1441 & 87 \\
    OPT-13B & summary & 320 & 240 \\
    OPT-30B & alpaca & 1348 & 159 \\
    OPT-30B & chatbot & 1379 & 191 \\
    OPT-30B & summary & 426 & 184 \\
    Llama-13B & alpaca & 1401 & 109 \\
    Llama-13B & chatbot & 1425 & 102 \\
    Llama-13B & summary & 234 & 252 \\
    Llama-32.5B & alpaca & 1274 & 254 \\
    Llama-32.5B & chatbot & 1250 & 242 \\
    Llama-32.5B & summary & 434 & 198 \\
    \bottomrule
  \end{tabular}
\end{table}

表\ref{Table:推理任务抢占行为记录}给出了序列最大输出长度为64时,不同框架推理过程中,平均每1000个请求对应的抢占行为次数。由表可知,AdaptiveLLM可以根据模型配置,灵活选择合适的内存优化策略。当序列最大输出长度限制在较低水平时,每个请求执行推理任务所需的迭代次数较少,资源需求量低,抢占鲜有发生,此时AdaptiveLLM和其他基准框架性能差距不大。随着最大输出长度的增加,有限的GPU内存无法满足需求,AdaptiveLLM调用基于开销感知的内存优化策略,展现性能优势。当最大输出长度过大时,无论是AdaptiveLLM还是基准框架,其批处理大小均限制在较低水平,但AdaptiveLLM仍具有明显优势(序列最大输出长度为256时,AdaptiveLLM相比于三个基准框架实现最高1.93$\times$、2.37$\times$和2.69$\times$的加速比)。

根据预测器给出的结果可知,对于大部分短序列(如Alpaca和Chatbot数据集)而言,张量交换开销小于张量重算开销。对于大部分长序列(如Summary数据集)而言,张量交换开销大于张量重算开销。

因此,在Chatbot和Alpaca数据集中,序列长度较短,批处理大小高。随着新token的生成,GPU显存无法满足众多用户请求的KV Cache存储需求,此时大量用户请求被换出到CPU中。当CPU内存不足时,则会强制执行张量重算。

对于Summary数据集而言,序列长度较长,批处理大小低。运行请求的KV Cache内存占用量增长缓慢,导致其推理过程中抢占现象发生相对较少,不会出现CPU内存不足的情况。此时,张量重算与张量交换次数能够真实反映开销预测值的比较结果。

\begin{figure*}[!htbp]
  \centering
  \includegraphics[width=0.9\linewidth]{平均带权周转时间.png}
  \caption{用户请求平均带权周转时间}
  \label{Fig:平均带权周转时间}
\end{figure*}

综上所述,在基于开销感知的内存优化策略下,AdaptiveLLM在GPU内存不足时预测张量交换与张量重算开销,并选择开销较小的内存优化技术执行,进而大幅度提升推理任务整体吞吐率。 

\subsection{公平性测试}

本文选取平均带权周转时间作为公平性测试指标。用户请求带权周转时间等于客户端响应延迟除以服务器端处理时长,如公式\ref{Eq:Weighted Around Time}所示。对于某一用户请求来说,$finish\_t$是其处理完毕时刻,$send\_t$是其从客户端发送至服务器端的时刻,$sche\_t$是其被AdaptiveLLM初次调度的时刻。平均带权周转时间($\geq 1$)越低,说明用户请求处理过程中的排队时间占比越低。 

\begin{equation}
  \begin{aligned}
    w\_around\_t = \frac{finish\_t - send\_t}{finish\_t - sche\_t}
  \end{aligned}
  \label{Eq:Weighted Around Time}
  \setlength{\abovedisplayskip}{0ex}
  \setlength{\belowdisplayskip}{2ex}
\end{equation}

图\ref{Fig:平均带权周转时间}展示了平均带权周转时间随单序列最大输出长度的变化情况。在相同数据集和LLM配置下,不同框架调度用户请求所产生的平均带权周转时间均随单序列最大输出长度的增加而增加。这是由于随序列长度的增加,有限的GPU内存所能存储的用户请求数量(即批处理大小)下降,导致后续请求的等待时间增加。Summary数据集相比于Alpaca和Chatbot数据集而言,相同条件下用户请求平均带权周转时间更高,但AdaptiveLLM仍能保持较大优势。

表\ref{Table:AdaptiveLLM相对于基准框架的平均带权周转时间缩减比}展示了单序列最大输出长度为64时,AdaptiveLLM相比于三个基准框架的平均带权周转时间缩减比。由表可知,AdapiveLLM的用户请求平均带权周转时间相比于vLLM下降了$20\%\sim50\%$,相比于vLLM$\_$s下降了$30\%\sim60\%$,相比于DeepSpeed-MII下降了$25\%\sim65\%$。

\begin{table}[H]
  \centering
  \caption{AdaptiveLLM相对基准框架的平均带权周转时间缩减比}
  \label{Table:AdaptiveLLM相对于基准框架的平均带权周转时间缩减比}
  \renewcommand{\arraystretch}{1.25}
  \small
  \begin{tabular}{c c c}
    \toprule
    \textbf{LLM} & \textbf{dataset} & \textbf{vLLM/vLLM\_s/DeepSpeed-MII} \\
    \midrule
    OPT-13B	& alpaca & 0.32$\times$/0.38$\times$/0.52$\times$ \\
    OPT-13B	& chatbot & 0.32$\times$/0.59$\times$/0.52$\times$ \\
    OPT-13B	& summary & 0.41$\times$/0.45$\times$/0.33$\times$ \\
    OPT-30B	& alpaca & 0.34$\times$/0.48$\times$/0.49$\times$ \\
    OPT-30B	& chatbot & 0.47$\times$/0.59$\times$/0.63$\times$ \\
    OPT-30B	& summary & 0.42$\times$/0.45$\times$/0.28$\times$ \\
    Llama-13B & alpaca & 0.26$\times$/0.39$\times$/0.54$\times$ \\
    Llama-13B & chatbot & 0.24$\times$/0.44$\times$/0.53$\times$ \\
    Llama-13B & summary & 0.41$\times$/0.43$\times$/0.43$\times$ \\
    Llama-32.5B & alpaca & 0.34$\times$/0.44$\times$/0.48$\times$ \\
    Llama-32.5B & chatbot & 0.44$\times$/0.56$\times$/0.51$\times$ \\
    Llama-32.5B & summary & 0.32$\times$/0.34$\times$/0.26$\times$ \\
    \bottomrule
  \end{tabular}
\end{table}

vLLM基于FCFS策略进行设计,在调度时优先考虑$swapped$队列,只有当$swapped$队列为空时才调度$waiting$队列,使得以交换方式被抢占的用户请求相比于以重算方式被抢占的用户请求,其重调度的优先级更高。结合4.4章节关于vLLM固定式抢占策略的分析可知,一部分用户请求被抢占后能够很快重新调度,而也有一部分用户请求被抢占后进入$waiting$队列的末尾,需要长时间等待。vLLM$\_$s和DeepSpeed同样缺少公平性考虑,因此他们在平均带权周转时间测试中均表现不佳。本文在AdaptiveLLM的设计中引入了基于公平性的用户请求调度策略,使得用户请求从客户端发送至服务器端后能够很快开始处理,不会出现长时间等待现象。

\subsection{预测误差测试}

\subsubsection{张量重算预测误差}

张量重算开销由张量重算开销分析器根据LLM模型层数、LLM模型隐藏维度、以及待处理token总数来预测得到。表\ref{Table:OPT模型单步迭代执行时间预测误差}和表\ref{Table:LLama模型单步迭代执行时间预测误差}分别展示了OPT模型和Llama模型单步推理执行时间的预测效果。OPT执行时间预测任务共有6.4w条训练数据和1.6w条测试数据,结果表明,随机森林回归模型性能最佳,其在拟合2次多项式时能够达到1.76\%的预测误差。Llama执行时间预测任务共有6.8条训练数据和1.7w条测试数据,结果表明,随机森林回归模型同样性能最佳,其在拟合2次多项式时能够达到1.30\%的预测误差。

\begin{table}[H]
  \centering
  \caption{OPT模型单步迭代执行时间预测误差}
  \label{Table:OPT模型单步迭代执行时间预测误差}
  \renewcommand{\arraystretch}{1.25}
  \small
  \begin{tabular}{c c c c c c}
    \toprule
    \textbf{模型-拟合次数} & \textbf{1} & \textbf{2} & \textbf{3} & \textbf{4} & \textbf{5} \\
    \midrule
    线性回归模型 & 46.52 & 46.65 & 28.75 & 11.86 & 9.32 \\ 
    决策树 & 1.81 & 1.81 & 1.81 & 1.81 & 1.81 \\ 
    随机森林 & 1.77 & 1.76 & 1.77 & 1.77 & 1.78 \\ 
    岭回归模型 & 46.52 & 46.37 & 28.45 & 11.51 & 7.36 \\ 
    lasso回归模型 & 40.22 & 25.53 & 27.38 & 26.08 & 25.49 \\ 
    弹性回归模型 & 111.89 & 123.62 & 91.67 & 87.59 & 86.48 \\ 
    梯度提升模型 & 15.57 & 16.05 & 14.80 & 15.09 & 14.68 \\ 
    KNN回归模型 & 2.55 & 2.80 & 2.89 & 3.00 & 3.05 \\ 
    \bottomrule
  \end{tabular}
\end{table}

\begin{table}[H]
  \centering
  \caption{LLama模型单步迭代执行时间预测误差}
  \label{Table:LLama模型单步迭代执行时间预测误差}
  \renewcommand{\arraystretch}{1.25}
  \small
  \begin{tabular}{c c c c c c}
    \toprule
    \textbf{模型-拟合次数} & \textbf{1} & \textbf{2} & \textbf{3} & \textbf{4} & \textbf{5} \\
    \midrule
    线性回归模型 & 76.41 & 69.44 & 39.61 & 12.91 & 9.18 \\ 
    决策树 & 1.33 & 1.32 & 1.33 & 1.33 & 1.34 \\ 
    随机森林 & 1.31 & 1.30 & 1.31 & 1.31 & 1.31 \\ 
    岭回归模型 & 76.41 & 69.01 & 39.18 & 12.73 & 7.72 \\ 
    lasso回归模型 & 69.23 & 33.57 & 34.42 & 35.16 & 31.58  \\ 
    弹性回归模型 & 127.18 & 139.7 & 100.18 & 94.94 & 93.51  \\ 
    梯度提升模型 & 22.42 & 21.97 & 19.42 & 19.99 & 19.38  \\ 
    KNN回归模型 & 2.24 & 2.36 & 2.48 & 2.63 & 2.68 \\ 
    \bottomrule
  \end{tabular}
\end{table}

\subsubsection{张量交换预测误差}

张量交换预测开销由张量交换开销分析器根据用户请求的KV Cache内存占用和GPU-CPU双向传输带宽而计算得到。本文针对模型Llama-13B和Llama-32.5B进行测试,其结果如图\ref{Fig:交换开销预测误差}所示。两个模型换入开销预测的MAPE误差分别为1.5\%和1.1\%,换出开销预测的MAPE误差分别为1.0\%和1.2\%。这说明本文提出的张量交换开销分析器实现了精准的开销预测。

\begin{figure}[!htbp]
  \centering
  \includegraphics[width=0.88\linewidth]{交换开销预测误差.png}
  \caption{交换开销预测误差}
  \label{Fig:交换开销预测误差}
\end{figure}

\subsection{开销测试}

基于开销感知的内存优化策略在获取张量重算和张量交换开销时,会带来新的预测开销。本文设计如下对照实验获取预测过程的开销:在吞吐率测试过程中,当GPU内存不足时调用开销比较过程,但最终使用vLLM提供的固定式内存优化策略(张量重算)。观察此情景下推理任务的总用时可知,预测开销在推理任务中仅占0.1\%至1\%。

基于公平性的用户请求调度策略也有一定的开销。理论上,该算法的时间复杂度为$O(r^2)$,其中$r$为running队列中用户请求的数量。在实际运行过程中,每次调度的开销在0.2ms左右,总开销在推理任务中仅占0.5\%至1\%。综上所述,AdaptiveLLM引入的两种优化策略所带来的额外开销均可忽略不计。

  \section{未来工作}

\subsubsection{面向截止时间点的调度}

用户请求往往需要在特定时间截点(DDL)前完成,DDL离当前时间点越近,该请求的紧迫程度就越高,在设计调度策略时应当有所考虑。Cilantro~\cite{Cilantro}等框架能够基于用户反馈对不同任务的优先级进行动态调整。而AdaptiveLLM的调度策略仅满足了公平性,没有考虑用户的真正需求。

\subsubsection{内存优化与前向传播的并行}

一、张量交换与前向传播的并行。张量交换的本质是GPU-CPU通信传输过程,而前向传播的本质是GPU计算过程。二者在传统模式下串行执行。AdaptiveLLM计划在内存优化决策器中设计一个交换线程和一个计算线程,并行完成两项任务,进一步减少张量交换带来的额外开销。

二、张量重算与前向传播的并行。SARATHI~\cite{SARATHI}框架研发了chunk-prefill技术,实现了prefill阶段与decode阶段的共置运行。由于张量重算的本质是prefill过程,因此若将该技术移植到AdaptiveLLM中,可以实现张量重算与前向传播的并行。

\subsubsection{张量并行与流水线并行} 

张量并行(Tensor Parallelism,TP)针对同一节点内的不同GPU实现;流水线并行(Pipeline Parallelism,PP)针对不同节点实现。目前,AdaptiveLLM的优化技术仅应用于单个GPU。实验平台提供了不同GPU间的PCIe通信和不同节点间的无线带宽网络通信,本文将在未来集成TP与PP并行技术。

另外,AdaptiveLLM提出的内存优化策略和用户请求调度策略在理论上能够扩展至大部分不同配置的服务器平台,将在未来开展进一步测试。


  \section{结论}
本文设计了AdaptiveLLM,一款基于张量交换和张量重算的LLM推理服务框架。AdaptiveLLM实现了张量重算开销预测与张量交换开销预测,其预测误差分别在2\%和4\%以下。AdaptiveLLM研发了基于开销感知的张量优化策略和基于公平性的用户请求调度策略。基于开销感知的张量优化策略用于在GPU内存不足时,执行开销较小的抢占方式来保证推理任务的顺利完成;基于公平性的用户请求调度策略则能够在GPU内存充足时重新调度被抢占的用户请求。实验表明,相比于vLLM框架,AdaptiveLLM有10\%-40\%的整体吞吐率提升,实现了服务器端的处理加速;且AdaptiveLLM能够以合理的方式调度用户请求,将平均带权周转时间优化为vLLM的60\%~80\%,减少等待时间,实现了面向客户端的实时请求处理。综上所述,AdaptiveLLM权衡整体吞吐率与单请求延时,化解二者在优化实现上的矛盾。
  \bibliography{references.bib}
\end{document}