\section{未来工作}

\subsubsection{面向截止时间点的调度}

AdaptiveLLM在调度策略上将进行进一步优化以支持面向截止时间点的调度。具体来说,用户向服务器端提交推理请求,往往期望在特定时间截点(DDL)前得到完整输出。DDL离当前时间越近,该任务的紧迫程度就越高,因此在调度时应赋予更高的优先级。对于在DDL前确定无法完成的请求,应直接返回客户端,避免浪费服务器资源。

\subsubsection{内存优化与前向传播的并行}

\begin{itemize}

    \item \textbf{张量交换与前向传播的并行。}张量交换的本质是GPU-CPU通信传输过程,而前向传播的本质是GPU计算过程。二者在传统模式下串行执行。AdaptiveLLM计划在内存优化决策器中设计一个交换线程和一个计算线程,并行完成两项任务,进一步减少张量交换带来的额外开销。

    \item \textbf{张量重算与前向传播的并行。}SARATHI~\cite{SARATHI}框架研发了chunk-prefill技术,实现prefill阶段与decode阶段共置运行。由于张量重算的本质是prefill过程,因此若将该技术移植到AdaptiveLLM中,可以实现张量重算与前向传播的并行。

\end{itemize}

\subsubsection{扩展至多机及多节点推理} 

AdaptiveLLM的优化技术目前仅支持单个GPU,在未来将扩展至多卡以及多节点推理。其中多卡推理一般需要结合张量并行(Tensor Parallelism,TP)。在TP场景下,由于模型隐藏维度被均分至不同GPU中,交换与重算开销的计算方式需要发生变化。多节点推理一般需要结合流水线并行(Pipeline Parallelism,PP)。在PP场景下,交换与重算开销的计算方式与单节点相同,AdaptiveLLM希望进一步将张量交换和张量重算填充至流水线气泡中以加速模型推理过程。

\subsubsection{扩展至更多硬件配置}

AdaptiveLLM所提出的内存优化策略与用户请求调度策略能够应用于不同配置的服务器硬件平台。迁移至新平台后,需要收集张量重算开销,训练单步推理时间预测模型,并获取GPU-CPU双向传输带宽信息。整个预处理过程通过数据收集脚本实现,运行时间不超过10min。本文在未来将针对预处理过程进行优化,以实现轻量级代码迁移。

