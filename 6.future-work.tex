\section{未来工作}
在未来一段时间内,本文将针对以下部分完善AdaptiveLLM的设计。 \par
\subsubsection{张量压缩技术}
张量在三级存储结构中的传输带来无法忽略的通信开销。由于三级存储结构间的传输带宽有限,因此随着LLM参数量和批处理大小的增加,通信开销成为主要性能瓶颈。压缩技术常与交换技术联合使用,通过矩阵变换等数学方式减少传输参数量。高效的压缩技术能够在不损失张量精度的前提下减小通信开销,进一步提升批处理大小上限,提升吞吐率。
\subsubsection{张量优化与前向传播的并行}
\textbf{张量交换}:张量交换的本质是GPU-CPU通信传输过程,而前向传播的本质是GPU计算。二者在传统模式下串行执行。AdaptiveLLM计划在内存优化决策器中设计一个交换线程和一个计算线程,并行完成两项任务,进一步减少张量交换带来的额外开销。 \par 
\textbf{张量重算}:SARATHI\cite{SARATHI}框架研发了chunk-prefill技术,实现prefill阶段与decode阶段的共置运行。由于张量重算的本质是Prefill阶段的执行,因此若将该技术移植到AdaptiveLLM中,则能够实现张量重算与前向传播的并行。
\subsubsection{张量并行与流水线并行\cite{Parallelism}} AdaptiveLLM目前仅针对张量并行度与流水线并行度均为1的场景进行优化,将在未来实现张量并行技术与流水线并行技术。